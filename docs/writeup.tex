\documentclass{article}
\usepackage{nips13submit_e,times}
\usepackage{hyperref}
\usepackage{url}

\title{User Interface Parsing}
\author{
Calvin Loncaric\\
Department of Computer Science and Engineering\\
University of Washington\\
Seattle, WA\\
\texttt{loncaric@cs.washington.edu}\\
\url{https://homes.cs.washington.edu/\~loncaric}}
\date{June 8, 2015}

\newcommand{\fix}{\marginpar{FIX}}
\newcommand{\new}{\marginpar{NEW}}

\nipsfinalcopy % Uncomment for camera-ready version

% Useful tips:
%
%    \verb+text+      verbatim text
%
% \begin{table}[t]
% \caption{Sample table title}
% \label{sample-table}
% \begin{center}
% \begin{tabular}{ll}
% \multicolumn{1}{c}{\bf PART}  &\multicolumn{1}{c}{\bf DESCRIPTION}
% \\ \hline \\
% Dendrite         &Input terminal \\
% Axon             &Output terminal \\
% Soma             &Cell body (contains cell nucleus) \\
% \end{tabular}
% \end{center}
% \end{table}
%
% \begin{figure}[h]
% \begin{center}
% %\framebox[4.0in]{$\;$}
% \fbox{\rule[-.5cm]{0cm}{4cm} \rule[-.5cm]{4cm}{0cm}}
% \end{center}
% \caption{Sample figure caption.}
% \end{figure}

\begin{document}
\maketitle

\begin{abstract}
The abstract paragraph should be indented 1/2~inch (3~picas) on both left and
right-hand margins. Use 10~point type, with a vertical spacing of 11~points.
The word \textbf{Abstract} must be centered, bold, and in point size 12. Two
line spaces precede the abstract. The abstract must be limited to one
paragraph.
\end{abstract}

\section{Introduction}

Intro goes here.

\section{Related Work}

For one: \cite{SketchREAD2007}.

\section{Framework}

\section{Result Discussion}

\begin{table}[h]
\begin{center}
\input{benchmark.tex}
\end{center}
\caption{Timings and output qualities for each of the example inputs. The
    timings were collected on a 2012 MacBook Pro with a 2-core 2.5 GHz Intel
    Core i5 processor and 8 Gb of RAM. The output quality is a qualitative
    measure manually assessed by the author for each file. ``Perfect'' means
    that the output layout perfectly matched the input sketch. ``Good'' means
    that all major layout elements were identified, but some measurements were
    off. ``Bad'' means that some layout elements were missing and/or many
    measurements were off. ``Fail'' means that the output did not resemble the
    input sketch.}
\label{tbl:benchmarks}
\end{table}


\section{Conclusion}

\bibliography{bibliography}{}
\bibliographystyle{unsrt}

\end{document}
